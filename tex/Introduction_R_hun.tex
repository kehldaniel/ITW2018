\documentclass[10pt]{beamer}

\usepackage[utf8]{inputenc}
\usepackage[T1]{fontenc}
\def\magyarOptions{frenchspacing=yes,refstruc=yes}
\def\bfdefault{b}

\usepackage[english,magyar]{babel}
\usepackage{graphics,epstopdf}
\usepackage{amsmath,amsfonts,amssymb,commath,bm}
\usepackage{ifthen}
\usepackage{helvet}
\usepackage{multicol}
\usepackage{multirow}
\usepackage{hyperref}
\usepackage[T1]{fontenc}
\usepackage{textcomp}
\usepackage{eurosym}
\usepackage{rotating}
\usetheme{ptektk}
\usepackage{xfrac}
\usepackage{wasysym}
\usepackage{listings}

\linespread{1.15}

\usecolortheme{orchid}
\useinnertheme{default}

\DeclareMathOperator*{\argmax}{arg\,max}
\DeclareMathOperator*{\argmin}{arg\,min}
\DeclareMathOperator{\Cov}{Cov}

\setlength{\parskip}{1ex}

\newcommand{\ind}{\perp\!\!\!\perp}
\newcommand*{\e}{\ensuremath{\mathrm{e}}}
\newcommand*{\T}{\ensuremath{\mathrm{T}}}
\newcommand*{\im}{\ensuremath{\mathrm{i}}}
\newcommand*{\sgn}{\ensuremath{\mathrm{sgn}}}
\newcommand*{\lin}{\ensuremath{\mathrm{lin}}}
\renewcommand*{\Re}{\ensuremath{\mathrm{Re}}}
\renewcommand*{\Im}{\ensuremath{\mathrm{Im}}}
\newcommand*{\E}{\ensuremath{\mathbf{E}}}
\renewcommand*{\P}{\ensuremath{\mathbf{P}}}

\newcommand{\bi}{\begin{itemize}}
\newcommand{\ei}{\end{itemize}}
\newcommand{\be}{\begin{enumerate}}
\newcommand{\ee}{\end{enumerate}}
\newcommand{\graybox}[1]{
	\par
	\fcolorbox{black}[rgb]{0.8,0.8,0.8}{
		\parbox{0.96\textwidth}{
			\vspace{0.5ex}
			\centering\parbox{0.93\textwidth}{
				#1
				}
			\vspace{0.5ex}
			}
		}
	\par
	}
	
\newcommand{\mybox}[1]{
	\par
	\fcolorbox{black}[rgb]{0.8,0.8,0.8}{
		\parbox{0.96\textwidth}{
			\vspace{0.5ex}
			\centering\parbox{0.93\textwidth}{
				#1
				}
			\vspace{0.5ex}
			}
		}
	\par
	}

\newtheorem{theo}{Tétel}
\newtheorem{lem}{Lemma}
\newtheorem{defi}{Definíció}

\title{Adatelemzés R-rel}
\author{Kehl Dániel}
\institute{PTE-KTK Közgazdasági és Ökonometria Intézet,\\Statisztika és Ökonometria Tanszék}
\date{2017/2018 ősz}

\begin{document}
\setbeamertemplate{background canvas}{\includegraphics[width=\paperwidth]{ktk_title}}
\begin{frame}
\titlepage
\end{frame}
\setbeamertemplate{background canvas}{\includegraphics[width=\paperwidth]{ktk_slide}}

\begin{frame}{Motiváció}
\begin{itemize}
\item az R világszerte, rengeteg területen használt nyelv/szoftver?
\begin{itemize}
\item társadalomtudományok
\item gazdaságtudományok, ökonometria
\item biometria: \textcolor{blue}{\url{http://www.bioconductor.org}}
\item $\dots$
\end{itemize}
\item ingyenes, open source
\item a felhasználó által szabadon bővíthető, legnagyobb előny is
\item egyik legjobb felhasználói közösség (r-help levlista)
\item sok nagyszerű könyv (R series of Wiley, UseR!), néhány ingyenes
\item talán egy kissé szubjektív történet: \textcolor{blue}{\url{http://r4stats.com/articles/popularity}}
\end{itemize}
\end{frame}

\begin{frame}{Tartalomjegyzék}
\tableofcontents
\end{frame}

\section{R alapok}

\begin{frame}{R alapok}
\framesubtitle{Az R Projektről}
\begin{itemize}
\item \textcolor{blue}{\url{http://www.r-project.org}}
\item az S nyelvből fejlődött ki
\item ingyenes
\item ún. csomagokból áll
\begin{itemize}
\item van néhány alap csomag (base) (az R telepítése után már rendelkezésre állnak)
\item továbbiak (11300 felett) találhatóak: CRAN (The \textcolor{red}{C}omprehensive \textcolor{red}{R} \textcolor{red}{A}rchive \textcolor{red}{N}etwork) -- topic views
\item egyéb felhasználói csomagok online és egyéb forrásokból
\end{itemize}
\item tudományos és céges életben is egyre elterjedtebb és elismertebb
\end{itemize}
\end{frame}

\begin{frame}{R alapok}
\framesubtitle{Pro és kontra}
\begin{itemize}
\item gyakorlatilag minden (statisztikai) módszer megtalálható (sőt)
\item jó minőségű, testreszabható, nyomtatáskész ábrák készítése
\item együttműködik más szoftverekkel (Excel, EViews, BUGS)
\item fejlesztői szabadság
\item a reproducible research támogatása
\item szkriptnyelv (korlátozott GUI és point-and-click funkciók)
\item egy elemzés eredménye nem néhány táblázat és ábra (mint sok más szoftver esetén), hanem a szükséges objektumok (amikkel aztán akár tovább számolhatunk, ábrázolhatunk)
\end{itemize}
\end{frame}

\begin{frame}{R alapok}
\framesubtitle{R Studio}
\begin{itemize}
\item népszerű IDE (integrated development environment)
\item ingyenes, kézreálló, kényelmes
\item ,,Matlabszerű''
\item letölthető \textcolor{blue}{\url{https://www.rstudio.com/ide/download/}}
\item néhány screenshot \textcolor{blue}{\url{https://www.rstudio.com/ide/screenshots/}}
\item egyre több kiegészítő szolgáltatás
\item ezen a kurzuson ezt fogjuk használni
\end{itemize}
\end{frame}

\begin{frame}{R alapok}
\framesubtitle{Alapfunkciók}
\begin{itemize}
\item prompt: > - inputot vár
\item demo(), a csomagokban jellemzően vannak demok, demo(persp), demo(graphics), demo(plotmath), demo(colors) etc.
\item használható számológépként, egyszerű operátorok és függvények
\item skalár változók definiálása: x = 1, x <- 1, 1 -> x (case sensitive!!!)
\item vagy assign(''x'', 1)
\item fel-le nyilakkal elérhetőek az előző parancsok
\item expression vs. assignment
\item ls() vagy objects(): a létező változók listája
\item rm() vagy rm(variablename): változó(k) eltávolítása
\item \# jelöli a kommentet a kódban
\end{itemize}
\end{frame}

\begin{frame}{R alapok}
\framesubtitle{Segítség!!!}
\begin{itemize}
\item ?fuggvenynev vagy help(fuggvenynev)
\begin{itemize}
\item első ránézésre a help túl bonyolultnak tűnhet, de némi tapasztalattal informatívak, könnyen használhatóak
\item a példán keresztül gyakran könnyebb megérteni a függvény működését (example(fuggvenynev))
\end{itemize}
\item help(''char'') és ?''char'' működik a speciális karakterek esetén
\item ?csomagnev
\item help.start()
\item help.search() vagy ??
\item R-intro.pdf az R-rel együtt telepítésre kerül
\item www, google, \textcolor{blue}{\url{http://stats.stackexchange.com}}
\item ha a fentiek nem működnek, r-help levlista
\begin{itemize}
\item előtte kötelező: \textcolor{blue}{\url{http://www.r-project.org/posting-guide.html}}
\end{itemize}
\end{itemize}
\end{frame}

\begin{frame}{R alapok}
\framesubtitle{Módok, típusok}
A legfontosabb változótípusok
\begin{itemize}
\item numeric (double precision)
\item integer, egész szám hozzárendelése: x <- 16L
\item complex, komplex szám hozzárendelése: y <- 5+2i
\item text, szöveg hozzárendelése: z <- ''mytext''
\item logical, logikai érték hozzárendelése: u <- TRUE, FALSE or T, F
\item stb (function, matrix, list, dataframe stb.)
\item mode() megadja egy objektum típusát
\item adott objektum struktúrájának ellenőrzése: str()
\end{itemize}
\end{frame}

\begin{frame}{R alapok}
\framesubtitle{Csomagok telepítése}
\begin{itemize}
\item install.packages(csomagnev)
\begin{itemize}
\item mirror választása, a szükséges fájlok letöltése CRAN-ról
\item ezt csak egyszer kell elvégezni (később frissítés)
\item install.packages(''googleVis'')
\end{itemize}
\item library(csomagnev): aktiválja az adott csomagot (library())
\begin{itemize}
\item innentől használhatóak a csomagban szereplő függvények, adatok
\item ezt minden esetben el kell végezni, ha szükség van a csomagra
\item library(,,googleVis'')
\end{itemize}
\item ?googleVis
\item érdemes kipróbálni: demo(WorldBank), demo(AnimatedGeoMap), néhány másodperc várakozásra számítani kell
\end{itemize}
\end{frame}

\begin{frame}{R alapok}
\framesubtitle{Házi feladat}
\begin{itemize}
\item R és RStudio telepítés
\item R-hez kötődő munkakörök keresése Magyarországon és a világban
\item Készítsen X változót, amely 5 értéket vesz fel!
\item Számítsa ki X négyzetét és gyökét!
\item Készítsen x változót, melynek értéke $3\pi$!
\item Számítsa ki a $X \cdot x$ szorzatot!
\item Írja felül x változót a $X \cdot x$ szorzattal!
\item futtassa az Appendix A parancsait az R-intro.pdf fájlból
\end{itemize}
\end{frame}

\section{A skalárokon túl}

\begin{frame}{A skalárokon túl}
\framesubtitle{Vektorok}
\begin{itemize}
\item általában egy skalárnál többre van szükségünk
\item c(): vektorok megadása az elemein keresztül
\item a vektor elemei mindig azonos típusúak
\begin{itemize}
\item y <- c(4, 3, 2.5, pi, 9)
\item x <- c(''abc'', ''de'')
\item z <- c(TRUE, FALSE, T, F)
\item y <- c(y, 3, 5)
\end{itemize}
\item vektor megadásának további módjai
\begin{itemize}
\item :: 1:5 1-től 5-ig álló számokból képez vektort
\item rep(): repetitions, ismétlések, lásd súgó!
\item seq(): összetettebb sorozatok
\item a fentiek kombinációja
\end{itemize}
\end{itemize}
\end{frame}

\begin{frame}{A skalárokon túl}
\framesubtitle{Recycling, fontos operátorok és függvények}
\begin{itemize}
\item (a vektorok) újrahasznosítás(a) fontos tulajdonság: a rövidebb (vektor) annyiszor újrahasznosításra kerül, ahányszor kell
\begin{itemize}
\item y/pi
\item y/c(1,2)
\end{itemize}
\item operátorok: +, -, *, /
\item matematikai műveletek: log, exp, sin, cos, sqrt
\item egyéb hasznos függvények: sum, length, mean, min/max, range, diff, cumsum, sd, var, summary
\item logikai operátorok: <, >, <=, >=, ==, !=, \& (és), | (vagy), ! (nem)
\item hiányzó értékek
\begin{itemize}
\item NA, minden ezzel végzett művelet eredménye NA, is.na()-val ellenőrzendő, NEM x == NA
\item NaN (0/0 stb., is.nan()), Not A Number, Inf a végtelen
\end{itemize}
\item paste függvény: szövegek összeolvasztása, lásd ?paste
\end{itemize}
\end{frame}

\begin{frame}{A skalárokon túl}
\framesubtitle{Házi feladat}
\begin{itemize}
\item képezze a következő vektorokat (anélkül, hogy begépelné őket)
\begin{itemize}
\item a: 1,3,5,7,$\dots$,99
\item b: 1,1,1,2,2,2,3,3,3
\item B: 1,2,3,1,2,3,1,2,3
\item d: 1,2,3,4,5,6,5,4,3,2,1
\item e: 1, 1/2, 1/3, 1/4, 1/5,$\dots$,1/10
\item f: 1, 8, 27, 64, 125, 216
\item g: 0, 25, 50, 75, $\dots$, 1000
\item h: ''x1'',''y2'',''z3'',''x4'',''y5'',''z6'' 
\end{itemize}
\item számítsa ki f harmonikus közepét f
\item számítsa ki e mértani közepét f
\item számítsa ki d sokasági szórását
\end{itemize}
\end{frame}

\begin{frame}{A skalárokon túl}
\framesubtitle{Vektorok elemeinek elérése, módosítása}
\begin{itemize}
\item az elemek elérése a [] operátorral
\begin{itemize}
\item a[2], a[2:5] a második és 2-5 elemeket adja vissza
\item a[-2] a második elemen kívül mindent visszaad
\end{itemize} 
\item logikai vektorok segítségével
\begin{itemize}
\item d < 3
\item d[d<3]
\end{itemize}
\item nevek megadásával
\begin{itemize}
\item names(b) <- letters[1:9]; b[''f'']
\end{itemize}
\item vektor elemeinek módosítása a [] operátorral
\begin{itemize}
\item a[2] <- ertek
\item a[2:5] <- ertekek
\item d[!is.na(d)] <- 0
\end{itemize}
\item lásd ?which, hasznos lehet
\end{itemize}
\end{frame}

\begin{frame}{A skalárokon túl}
\framesubtitle{Mátrixok}
\begin{itemize}
\item gyakorlatilag ugyanaz, mint a vektor, két dimenzióval
\begin{itemize}
\item M <- matrix(c(1,2,3,4),2,2)
\item az alapértelmezés az oszloponkénti feltöltés
\end{itemize}
\item hasznos fgvek: t, diag, dim, nrow, ncol, cbind, rbind
\item elemek elérése a vektorokéhoz hasonló M[3,] vagy M[3,5]
\item mátrix-szorzás: \%*\%
\item lásd solve és crossprod függvények
\item az általánosított, többdimenziós struktúrákat array-nek nevezzük X[ , , ]
\item mátrixokban és arrayben az értékek módosítása a vektorokéhoz hasonló
\end{itemize}
\end{frame}

\begin{frame}{A skalárokon túl}
\framesubtitle{Factor, list}
A factor a minőségi ismérvek tárolására szolgál
\begin{itemize}
\item f <- factor(rep(letters[1:10], 3))
\item levels(f)
\end{itemize}
A list egy (akár) különböző típusú, de valamiért összetartozó elemeket foglal magában (pl. egy függvény outputja, stb)
\begin{itemize}
\item mylist <- list(e,f,g)
\item list elemeinek elérése: mylist[[1]]
\item mylist[1] egy egy hosszúságú listát ad vissza
\item VAGY
\item mylist2 <- list(res1=e,res2=f,res3=g)
\item ekkor az elemek elérése: mylist2\$res1
\end{itemize}
\end{frame}

\begin{frame}{A skalárokon túl}
\framesubtitle{Data frame}
\begin{itemize}
\item a data framere gondolhatunk úgy, mint egy speciális listára, aminek az elemei azonos hosszúságú vektorok
\item vagy úgy is mint egy speciális mátrixra, ami különböző típusú oszlopokat (numeric, logical, factor) tartalmaz(hat)
\item illetve úgy is, mint egy ,,hagyományos'' adatállományra Excelből vagy SPSS-ből
\item az adatelemzés igáslova R-ben, sorai megfigyelések, oszlopai változók
\item a <- c(''John'', ''Karl'', ''Kate'')
\item b <- c(4.5, 4.8, 3.2)
\item d <- factor(c(''male'',''male'',''female''))
\item df <- data.frame(statpoint=b,gender=d); rownames(df) <- a
\item próbáld ki: df\$b, df\$statpoint, names(df) and str(df)
\end{itemize}
\end{frame}

\begin{frame}{A skalárokon túl}
\framesubtitle{Házi feladat}
\begin{itemize}
\item Hozzon létre egy vektort, melyben 1-től 125-ig az egész számok vannak!
\item Alakítsa át egy 5 oszlopot és 25 sort tartalmazó mátrixszá, a számokat soronként használja fel!
\item Mi a 3. sor, 2. oszlop értéke?
\item Változtassa meg a 10. sor értékeit csupa 0-ra!
\item Alakítsa át az adatokat egy 5x5x5-ös arrayre!
\item Készítsen egy data framet a családja tagjairól, legalább 4-5 megfigyeléssel és 5-6 változóval, amik különböző típusúak legyenek!
\end{itemize}
\end{frame}

\section{Ciklusok, saját függvények}

\begin{frame}{Ciklusok, saját függvények}
\framesubtitle{for, if, while és barátaik}
A saját függvény, vagy a programkód lefutása szabályozható
\begin{itemize}
\item for (i in 1:100)\{print(2*i)\}
\item for ciklusok egymásba ágyazhatóak, pl. egy mátrix elemeinek végiglátogatására
\item a ciklus megadására sok lehetőség van, lásd help
\item if(x == 3)\{y<-10\} else \{y<-20\}
\item i<-0; while(i<10) \{print(i);i<-i+1\}
\end{itemize}
\end{frame}

\begin{frame}{Ciklusok, saját függvények}
\framesubtitle{Saját függvények készítése}
\begin{itemize}
\item az R bármikor kiegészíthető saját függvényekkel
\end{itemize}
fuggvenynev <- function()\{\\
tetszoleges muveletek\\
return(eredmenyek)
\}
\begin{itemize}
\item írjon egy függvényt, ami megduplázza egy vektor elemeit (ez a megoldás nem ellenőrzi x-et)
\end{itemize}
doublebouble <- function(x)\{\\
2*x\\
\}
\begin{itemize}
\item írjon egy függvényt, ami egy numerikus vektor alapján az átlagbecslés standard hibáját határozza meg
\end{itemize}
\end{frame}

\section{Adatok beolvasása}

\begin{frame}{Adatok beolvasása}
\framesubtitle{Beépített adatok, helyi fájlok}
\begin{itemize}
\item R-data.pdf ad segítséget
\item edit(), fix(), newx <- edit(data.frame())
\item beépített adatállományok különböző csomagokban data()
\begin{itemize}
\item cars adatállomány
\item names(cars)
\item cars\$speed
\item attach(), detach() (már nem használatos)
\item data(package = .packages(all.available = TRUE))
\end{itemize}
\item külső adatállományok (working directory - getwd(), setwd())
\begin{itemize}
\item scan()
\item read.table() vagy read.csv(), file=file.choose()
\item foreign csomag: egyéb szoftverek, lásd ?foreign
\item xlsx package
\end{itemize}
\end{itemize}
\end{frame}

\begin{frame}{Adatok beolvasása}
\framesubtitle{Külső adatforrások}
\begin{itemize}
\item online adatok
\begin{itemize}
\item read.table("http://lib.stat.cmu.edu/jcgs/tu",skip=4,header=T)
\end{itemize}
\item adatbázisok
\begin{itemize}
\item \url{https://CRAN.R-project.org/package=RMySQL}
\item \url{https://CRAN.R-project.org/package=RODBC}
\item \url{https://CRAN.R-project.org/package=ROracle}
\item \url{https://db.rstudio.com/}
\end{itemize}
\item web service-n keresztül
\begin{itemize}
\item \url{https://CRAN.R-project.org/package=WDI}
\item \url{https://CRAN.R-project.org/package=weatherData}
\item \url{https://CRAN.R-project.org/package=Quandl}
\item stb.
\end{itemize}
\end{itemize}
\end{frame}

\begin{frame}{Adatok beolvasása}
\framesubtitle{Feladatok}
\begin{itemize}
\item olvassa be egy dataframe-be a hallgatok.xlsx fájlban található adatokat!
\item olvassa be a mutatok.csv fájlt!
\item olvassa be az LTE423.csv fájlt!
\begin{itemize}
\item ügyeljen az időpontok feldolgozására
\item alakítsa át a szélességi és hosszúsági fok adatokat
\item pro tipp: setClass
\item írjon függvényt, ami a hasonló struktúrájú fájlokat képes beolvasni és dataframe-t ad vissza
\end{itemize}
\item keressen olyan R csomagokat, melyekkel adatok tölthetőek le könnyedén
\end{itemize}
\end{frame}

\section{Leíró statisztika, ábrák}

\begin{frame}{Leíró statisztika, ábrák}
\framesubtitle{Néhány alapvető leíró függvény}
\begin{itemize}
\item használjuk a Salaries adatállományt a car csomagból
\item ?str
\item ?summary
\item minőségi ismérvek
\begin{itemize}
\item table(): gyakorisági/kontingencia tábla
\item barplot(), pie(), dotchart()
\end{itemize}
\item mennyiségi ismérvek
\begin{itemize}
\item stem(), hist(), boxplot()
\item mean(), sd(), var(), quantile(), IQR()
\end{itemize}
\end{itemize}
\end{frame}

\begin{frame}{Leíró statisztika, ábrák}
\framesubtitle{Két vagy több változó}
\begin{itemize}
\item cor(), cov()
\item table(), margin.table(), addmargins(), prop.table()
\item barplot(), qqplot(), boxplot()
\item plot() [plot(0:25,pch=0:25)]
\item anscombe adatállomány ?anscombe
\item több változó: example(pairs)
\item Chernoff-arcok
\end{itemize}
\end{frame}

\begin{frame}{Leíró statisztika, ábrák}
\framesubtitle{Három alapvető grafikus megközelítés}
\begin{itemize}
\item basic plot: a legegyszerűbb használni, lépésről lépésre felépített ábrák, hozzáadhatunk pontokat, egyenest, címet, szöveget stb.
\item lattice (nem ismerem, nem használtam)
\item ggplot2: nagyon népszerű, sajátos logika
\begin{itemize}
\item Grammar of Graphics
\item \textcolor{blue}{\url{http://ggplot2.org}}
\item aesthetics, geoms, qplot(), ggplot()
\item qplot(displ,hwy,data=mpg)
\item qplot(displ,hwy,data=mpg,color=drv)
\item qplot(displ,hwy,data=mpg,geom=c("point",	"smooth"))
\item qplot(hwy,data=mpg,fill=drv)
\end{itemize}
\end{itemize}
\end{frame}

\begin{frame}{Leíró statisztika, ábrák}
\framesubtitle{Néhány R-ben készített ábra}
\begin{itemize}
\item \textcolor{blue}{\url{http://www.stanford.edu/~cengel/cgi-bin/anthrospace/wp-content/uploads/2011/02/ddMapCHAS.png}}
\item \textcolor{blue}{\url{http://robjhyndman.com/Rfiles/animation/animation.gif}}
\item \textcolor{blue}{\url{http://ryouready.files.wordpress.com/2010/03/playwith_demo.png}}
\item \textcolor{blue}{\url{http://spatialanalysis.co.uk/category/r-spatial-data-hints}}
\item \textcolor{blue}{\url{http://www.statmethods.net/advgraphs/index.html}}
\item \textcolor{blue}{\url{http://shiny.rstudio.com/gallery/}}
\end{itemize}
\end{frame}

\begin{frame}{Leíró statisztika, ábrák}
\framesubtitle{Házi feladat -- Olvassa be a hallgatok.xlsx fájlt}
\begin{itemize}
\item nevezze át a változókat kódolásra alkalmas névre
\item készítsen egy summary-t a dataframe-re
\item készítsen hisztogramot az összes jövedelem változó alapján, készítsen magyar címet, feliratokat, az y tengelyen százalékos értékek szerepeljenek
\item mekkora a hobbira költött összeg átlaga és szórása? értelmezze is a kapott eredményeket!
\item készítsen kereszttáblát a munka jellege és a településtípus változók segítségével! hány diákmunkás van a megyeszékhelyen lakók között?
\item számítsa ki sporttevékenységenként a sportra költött összeg átlagát!
\item készítsen pontdiagramot a sporttal töltött idő és a sportra költött összeg változókból! mit tapasztal?
\end{itemize}
\end{frame}

\section{Következtető statisztika}

\begin{frame}{Következtető statisztika}
\framesubtitle{Várható érték tesztek a gyakorlatban}
\begin{itemize}
\item egymintás próbák
\begin{itemize}
\item t-próba a várható értékre
\item nem-paraméteres próbák
\end{itemize}
\item  kétmintás próbák
\begin{itemize}
\item független mintás próbák a várható értékre
\item párosított próbák a várható értékre
\item nem-paraméteres próbák
\item asszociációs kapcsolat vizsgálata
\end{itemize}
\item három, vagy több minta
\begin{itemize}
\item  ANOVA
\end{itemize}
\item illeszkedésvizsgálat (normalitás teszt)
\end{itemize}
\end{frame}

\begin{frame}{Következtető statisztika}
\framesubtitle{Gyakorló feladat}
\begin{itemize}
\item Nyisson meg egy tetszőleges beépített, vagy máshonnan beolvasott adatállomány!
\item Tegyen fel kérdéseket, melyeket a hipotézis ellenőrzés egyszerű eszközeivel vizsgálni tud!
\item Készítsen egy-egy ábrát, amely segíthet a kérdések eldöntésében, illetve végezze el a vonatkozó teszte(ke)t!
\item Írja az ábrára a vonatkozó teszt eredményét!
\end{itemize}
\end{frame}

\begin{frame}{Következtető statisztika}
\framesubtitle{Korreláció}
\begin{itemize}
\item Anscombe dataset
\item Számítsa ki a változók átlagát, szórását, x és y közötti korrelációt!
\item Készítsen egy-egy ábrát!
\item Írja az ábrára a vonatkozó teszt eredményét!
\item korrelációs mátrix számítása
\item pairs() függvény (ggplot)
\end{itemize}
\end{frame}

\section{ANOVA}

\begin{frame}{ANOVA}
\framesubtitle{Variancia-analízis}
\begin{itemize}
\item ANalysis Of VAriance
\item három vagy több átlag egyezősége
\item számítsuk ki az átlagot fajonként az iris adatállományban
\item szórást is!
\begin{itemize}
\item tapply(iris\$Sepal.Length,iris\$Species,mean)
\item tapply(iris\$Sepal.Length,iris\$Species,function(x) c(mean(x), sd(x)))
\item úgy tűnik, hogy szignifikáns különbségek vannak
\item tapply és társai sokszor hasznosak
\end{itemize}
\item ANOVA futtatása
\begin{itemize}
\item myfirstanova <- aov(Sepal.Length $\sim$ Species, data=iris)
\end{itemize}
\item myfirstanova egy új objektum, ami a fontos eredményeket tartalmazza
\begin{itemize}
\item summary(myfirstanova)
\item names(myfirstanova)
\end{itemize}
\end{itemize}
\end{frame}

\section{Lineáris regresszió}

\begin{frame}{Kétváltozós lineáris regresszió}
\framesubtitle{A modell}

\[
\mathbf{E}\left(Y \mid X =x\right) = \beta_0 + \beta_1 x
\]
ahol $\beta_0$ és $\beta_1$ a lineáris egyenes két paramétere, vagy
\[
Y_i = \beta_0 + \beta_1 x + e_i
\]
ahol $e_i$ a véletlen hiba 0 várható értékkel és $\sigma^2$ varianciával.

A sokaság nem ismert, csupán $x_i$ és $y_i$ párokból álló minta áll rendelkezésre, ami alapján a ,,legjobb'' egyenes meghatározása a cél. Az ezzel előrejelzett értéket $\hat{y}_i$ jelöli.
\end{frame}

\begin{frame}{Kétváltozós lineáris regresszió}
\framesubtitle{A modell -- LNM}

A minta alapján meghatározott 
\[
\hat{y}_i = b_0 + b_1 x_i
\]
egyenes a lehető legközelebb helyezkedik el a megfigyelt $y_i$ értékekhez, a kettő közötti $y_i - \hat{y}_i$ különbséget reziduumnak ($\hat{e}_i$) nevezzük.

A legjobb egyenes megtalálása jellemzően a legkisebb négyzetek módszerével történik.
\[
\text{RSS} = \sum_{i = 1} ^ n \left(y_i - \hat{y}_i \right)^2 = \sum_{i = 1} ^ n \left(y_i - b_0 - b_1 x_i \right)^2 \to \text{min}
\]
\end{frame}

\begin{frame}{Kétváltozós lineáris regresszió}
\framesubtitle{A modell -- becsült paraméterek}
\[
\hat{\beta_0} =  \overline{y} - \hat{\beta_1} \overline{x}
\]

\[
\hat{\beta_1} =  \dfrac{\sum \left(x_i - \overline{x}\right)\left(y_i - \overline{y}\right)}{\sum \left(x_i - \overline{x}\right)^2}
\]
illetve becsülnünk kell a hibatag varianciáját ($\sigma^2$)
\[
S^2 = \dfrac{1}{n-2} \sum \hat{e}_i^2
\]
\end{frame}

\begin{frame}{Kétváltozós lineáris regresszió}
\framesubtitle{Következtetés a paraméterekkel kapcsolatban -- feltevések}
A paramétereket mintából becsültük, de a sokasági paraméterekre vagyunk kíváncsiak, a következtetéshez szükséges feltételek:
\begin{itemize}
\item a két változó között lineáris a kapcsolat
\item az $e_1, e_2, \dots, e_n$ hibatagok egymástól függetlenek
\item az $e_1, e_2, \dots, e_n$ hibatagok varianciája állandó $\sigma^2$
\item a hibatagok normális eloszlást követnek 0 várható értékkel és $\sigma^2$ varianciával, azaz $e \mid X \sim \mathcal{N}\left(0, \sigma^2 \right)$
\end{itemize}
A fenti feltételek teljesülésének vizsgálatára a későbbiekben visszatérünk (modell diagnosztika)
\end{frame}

\begin{frame}{Kétváltozós lineáris regresszió}
\framesubtitle{Következtetés a paraméterekkel kapcsolatban -- hipotézis ellenőrzés}
Belátható, hogy a
\[
T = \dfrac{\hat{\beta}_j^0 - \beta_j}{se\left(\hat{\beta}_j \right)}
\]
próbafüggvény $n-2$ szabadságfokú t-eloszlást követ $H_0$ alatt, amivel tehát tesztelhető a $H_0:\beta_j = \beta_j^0$ hipotézis, ahol $se\left(\hat{\beta}_j \right)$ a $\beta$ paraméterhez tartozó becsült standard hiba.

A leggyakrabban végzett hipotézis $\beta_j^0 = 0$-ra vonatkozik. A statisztikai szoftverek az erre vonatkozó p-értékeket automatikusan közlik valamennyi paraméterre.
\end{frame}

\begin{frame}{Kétváltozós lineáris regresszió}
\framesubtitle{Következtetés a paraméterekkel kapcsolatban -- konfidencia intervallum}
A paraméterek pontbecslése minta alapján készül, a rájuk vonatkozó konfidencia intervallum 
\[
\hat{\beta}_j \pm t_{\alpha/2}^{n-2}se\left(\hat{\beta}_j \right)
\]
módon számítható, ahol $t_{\alpha/2}^{n-2}$ az $n-2$ szabadságfokú t-eloszlás kvantilise, ami $\alpha/2$ területet vág le felfelé.

A konfidencia intervallum és a hipotézisvizsgálat szorosan összetartozik.
\end{frame}

\begin{frame}{Kétváltozós lineáris regresszió}
\framesubtitle{Következtetés a regressziós egyenesre vonatkozóan}
Gyakran vagyunk kíváncsiak arra, hogy a (sokasági) regressziós egyenes milyen értéket vesz fel egy adott $X=x^*$ helyen, azaz
\[
\mathbf{E}\left(Y \mid X = x^* \right) = \beta_0 + \beta_1 x^*
\]
feltételes várható értékre vagyunk kíváncsiak.

Belátható, hogy a szokásos jelölések mellett a konfidencia intervallum
\[
\hat{\beta}_0 + \hat{\beta}_1 x^* \pm t_{\alpha/2}^{n-2}S\sqrt{\dfrac{1}{n}+\dfrac{\left(x^* - \overline{x}\right)^2}{\sum \left(x_i - \overline{x}\right)^2}}
\]
\end{frame}

\begin{frame}{Kétváltozós lineáris regresszió}
\framesubtitle{Előrejelzés}
Amennyiben $X$ adott értékénél az $Y$ tényleges értékére vagyunk kíváncsiak (nem $Y$ várható értékére, mint az előző esetben), az $Y$-on  belüli véletlen ingadozást is figyelembe kell vennünk, azaz az előrejelzési intervallum szélesebb kell hogy legyen.

Az ún. prediction interval $Y^*$-ra vonatkozóan
\[
\hat{\beta}_0 + \hat{\beta}_1 x^* \pm t_{\alpha/2}^{n-2}S\sqrt{1 + \dfrac{1}{n}+\dfrac{\left(x^* - \overline{x}\right)^2}{\sum \left(x_i - \overline{x}\right)^2}}
\]
\end{frame}

\begin{frame}{Kétváltozós lineáris regresszió}
\framesubtitle{Regressziós egyenes vizsgálata}
A későbbiekben több $X$ változó esetén kíváncsiak lehetünk, hogy valamennyi $\beta$ paraméter egyenlő-e nullával, azaz a lineáris modellünk releváns-e.
Ezt egy F-próbával végezzük, a próba neve variancia-analízis,  nullhipotézise, hogy minden sokasági paraméter 0.

Kétváltozós esetben a $\beta_1$-re vonatkozó t-próba próbafüggvény értéke és az F-próba próbafüggvény értéke négyzetes kapcsolatban állnak.

Az $R^2$ mutató a modell illeszkedésének jóságát mutatja meg, többváltozós esetben foglalkozunk vele, kétváltozós esetben a korrelációs együttható négyzete.
\end{frame}

\section{Modell diagnosztika}

\begin{frame}{Modell diagnosztika}
\framesubtitle{Reziduumok vizsgálata}
A reziduumok vizsgálata vizuális segítséget ad a korábban bemutatott modell-feltevések ellenőrzéséhez. Amennyiben nem fedezhető fel különösebb mintázat az $x$ értékek és reziduumok ábrázolásával nyert ábrá(ko)n, az a modell megfelelőségét támasztja alá.

További eszközök:
\begin{itemize}
\item olyan pontok azonosítása, melyek nagy hatással vannak a becsült paraméterekre (leverage (befolyásoló) pontok, outlierek)
\item a hibatag konstans varianciájának feltevése
\item a hibatag normalitásának vizsgálata (elsősorban kis minta esetén)
\end{itemize}
\end{frame}

\begin{frame}{Modell diagnosztika}
\framesubtitle{Befolyásoló pontok}
A befolyásoló pontok nagy hatással vannak a becsült paraméterekre,
\begin{itemize}
\item messze kell lenniük a többi értéktől $x$ tekintetében
\item nem illenek a többi pont által leírt mintába (,,jó'' és ,,rossz'' leverage pontok)  
\end{itemize}
Az $i$. megfigyelés befolyásoló (leverage) értéke az alábbi formula
\[
h_{ii}=\dfrac{1}{n}+\dfrac{\left(x_i-\overline{x}\right)^2}{\sum_j \left(x_j-\overline{x}\right)^2}
\]
Az átlagos $h_{ii}$ érték $\frac{2}{n}$ értéket vesz fel, ettől jóval nagyobb leverage érték esetén jelentős az adott pont hatása. $\to$ kihagyás, eltérő modell
\end{frame}

\begin{frame}{Modell diagnosztika}
\framesubtitle{Standardizált reziduumok}
A reziduumok vizsgálata félrevezető lehet, ha a mintában nagy leverage értékű pont(ok) találhatóak, mert az ezekhez tartozó reziduumok varianciája kicsi lesz, akkor is, ha a hibatag varianciája konstans.

Az $i$. megfigyeléshez tartozó standardizált reziduum:
\[
r_i = \dfrac{\hat{e}_i}{s\sqrt{1-h_{ii}}}
\]
A standardizált reziduum ábrák tehát azt mutatják meg, hogy az adott pont hány becsült szórásnyira van az illesztett regressziós egyenestől, amennyiben messze, outliernek, kiugró értéknek tekintjük.
\end{frame}

\begin{frame}{Modell diagnosztika}
\framesubtitle{Cook távolság}
Amennyiben egy-egy pont jelentősen befolyásolja a modellünket, megvizsgálhatjuk, hogy kihagyása mennyivel módosítja a többi ponthoz tartozó becslésünket. Belátható, hogy
\[
D_i = \dfrac{r_i^2}{2}\dfrac{h_{ii}}{1-h_{ii}}
\]
amely tehát egyszerre méri adott pont befolyását és kiugró voltát.

Hüvelykujj-szabályként a $\frac{4}{n-2}$ értéket ajánlják annak eldöntésére, hogy a Cook távolság magasnak tekintendő-e.
\end{frame}

\begin{frame}{Modell diagnosztika}
\framesubtitle{Normalitás, konstans variancia}
A hibatag normalitása fontos, elsősorban az előrejelzésekre vonatkozó intervallumok (prediction interval) esetén, illetve kis minták esetén a paraméterekre vonatkozó konfidencia intervallumok esetén is. A feltételt leggyakrabban a standardizált reziduumok eloszlása segítségével vizsgáljuk (pl. QQ-plot)

A hibatag nem-konstans varianciája (heteroszkedaszticitás) minden következtetéses statisztikai eszközt bizonytalanná tesz, vizuálisan jellemzően a standardizált reziduumokat, vagy az abszolút értékük gyökét vizsgáljuk, illetve léteznek statisztikai próbák (pl. Breush-Pagan test) $\to$ transzformációk
\end{frame}

\begin{frame}{Modell diagnosztika}
\framesubtitle{Transzformációk}
Transzformációkat gyakran alkalmaznak az ökonometriában
\begin{itemize}
\item a nem konstans variancia kezelésére
\item a nem linearitás kezelésére
\end{itemize}
A leggyakrabban használt transzformációk (de a lehetőségek száma nagyon nagy)
\begin{itemize}
\item gyökvonás
\item logaritmus
\item Box-Cox transzformáció (hatvány transzformáció, célja a normalitáshoz minél közelebb kerülni)
\end{itemize}
\end{frame}

\section{Többváltozós lineáris regresszió}

\begin{frame}{Többváltozós lineáris regresszió}
\framesubtitle{Alapmodell}
A modell a kétváltozós regresszióhoz nagyon hasonló, több magyarázó változóval
\[
Y_i = \beta_0 + \beta_1x_{1i}+ \beta_2x_{2i} + \dots + + \beta_px_{pi}+e_i
\]
ahol $p$ a magyarázó változók száma. Mátrix formában
\[
\mathbf{Y}=\mathbf{X}\beta + \mathbf{e}
\]
A továbbiakban mátrixalgebra (és képletek) nélkül.
\end{frame}

\begin{frame}{Többváltozós lineáris regresszió}
\framesubtitle{Modell szelekció}
A többváltozós regresszió esetén felmerül annak a kérdése, hogy melyik potenciális magyarázó változókat használjuk a modellünkben.

A megismert $R^2$ mutató monoton növekvő újabb változók bevonásával, illetve a korrigált $R^2$ mutató is hajlamos túl sok változót javasolni.

Alternatívaként az ún. információs kritériumok szolgálhatnak (AIC, AICc, BIC).
\end{frame}

\begin{frame}{Többváltozós lineáris regresszió}
\framesubtitle{Diagnosztika újdonság}
A korábbi eszközök mellett egy újabb problémára kell figyelnünk modellezés esetén, méghozzá a magyarázó változók közötti kapcsolatra.

Amennyiben a magyarázó változók között szoros (lineáris) kapcsolat van, a becsült paraméterek nagyságrendje, vagy akár előjele megbízhatatlanná válik, mert a becslés varianciája nagymértékben megnövekedik (inflálódik). A probléma detektálása a VIF (variancia infláló faktor) segítségével végezhető el. A multikollinearitás kezelése jelen tananyagunkat meghaladja.
\end{frame}

\section{További hasznos csomagok}

\begin{frame}{tidyverse}
\framesubtitle{rövid bemutatás}
\begin{itemize}
\item csomagok összessége
\begin{itemize}
\item ggplot2, dplyr, purrr, tidyr, readr stb.
\end{itemize}
\item új megközelítés adatkezeléshez
\item Hadley Wickham az egyik fő fejlesztő
\item \url{http://r4ds.had.co.nz/}
\item ,,pipe'' \%>\% (magrittr csomag) (CTRL+SHIFT+m)
\item olvashatóbb, tisztább, logikusabb (sokszor rövidebb) kód
\item f(g(x)) helyett x \%>\%g() \%>\% f()
\item feladat: olvassa be az LTE423.csv fájlt, szűrje le a 100 méter feletti magasságú pontokat majd számoljon mutatókat a változókra egy sorban (gyakorlásként tidyverse nélkül is)
\end{itemize}
\end{frame}

\begin{frame}{tidyverse}
\framesubtitle{rövid bemutatás}
\begin{itemize}
\item csomagok összessége
\begin{itemize}
\item ggplot2, dplyr, purrr, tidyr, readr stb.
\end{itemize}
\item új megközelítés adatkezeléshez
\item Hadley Wickham az egyik fő fejlesztő
\item \url{http://r4ds.had.co.nz/}
\item ,,pipe'' \%>\% (magrittr csomag) (CTRL+SHIFT+m)
\item olvashatóbb, tisztább, logikusabb (sokszor rövidebb) kód
\item f(g(x)) helyett x \%>\%g() \%>\% f()
\item feladat: olvassa be az LTE423.csv fájlt, szűrje le a 100 méter feletti magasságú pontokat majd számoljon mutatókat a változókra egy sorban (gyakorlásként tidyverse nélkül is)
\end{itemize}
\end{frame}

\begin{frame}{tidyverse}
\framesubtitle{rvest}
\begin{itemize}
\item ,,harvest'' adatokat weboldalakról
\item feladat: adatok letöltése \textit{\href{https://www.transfermarkt.com/premier-league/einnahmenausgaben/wettbewerb/GB1/plus/0?ids=a&sa=&saison_id=2012&saison_id_bis=2012&nat=&pos=&altersklasse=&w_s=&leihe=&intern=0}{innen}}
\item feladat2: más bajnokságok és időszakok
\end{itemize}
\end{frame}

%\begin{frame}{ReporteRs}
%\framesubtitle{MS dokumentumok generálása R-ből	}
%\begin{itemize}
%\item ,,harvest'' adatokat weboldalakról
%\item feladat adatok letöltése
%\item url{https://www.transfermarkt.com/premier-league/einnahmenausgaben/wettbewerb/GB1/plus/0?ids=a&sa=&saison_id=2012&saison_id_bis=2012&nat=&pos=&altersklasse=&w_s=&leihe=&intern=0}
%\item feladat2: más bajnokságok és időszakok
%\end{itemize}
%\end{frame}

%\section{Simulation}
%\begin{frame}{Simulation}
%\framesubtitle{Random number generator}
%\begin{itemize}
%\item ,,Repetitio est mater studiorum''
%\item R handles most of the distributions with the help of 4 functions
%\begin{itemize}
%\item d\textit{distname}: pdf value (ie. dnorm)
%\item p\textit{distname}: cdf value (ie. pnorm)
%\item r\textit{distname}: random number generator (ie. rnorm)
%\item q\textit{distname}: quantile function (ie. qnorm)
%\end{itemize}
%\end{itemize}
%\begin{footnotesize}
%\begin{center}
%\begin{tabular}{rc|rc}
%    distribution & \textit{distname} &distribution & \textit{distname}\\
%    \hline
%    beta  & beta  & chi-squared & chisq \\
%    binomial & binom & lognormal & lnorm \\
%    Cauchy & cauchy & neg. binomial & nbinom \\
%    uniform & unif  & normal & norm \\
%    exponential & exp   & Poisson & pois \\
%    gamma & gamma & Student-t & t \\
%\end{tabular}%
%\end{center}
%\end{footnotesize}
%Try recreating the standard normal table at the back of your stat book!
%\end{frame}
%
%\begin{frame}{Simulation}
%\framesubtitle{Exercise}
%Simulate $REPS=1000$ random vectors of size $n=100$ from a non-normal distribution! Choose different distributions!
%\begin{itemize}
%\item Draw a histogram of one of the samples!
%\item Calculate the mean of all 1000 samples!
%\item Draw a histogram of the sample means!
%\item What are your experiences? Which is the theorem applicable here?
%\item Try to change $REPS$ and $n$! What are your results?
%\end{itemize}
%\end{frame}
\end{document}